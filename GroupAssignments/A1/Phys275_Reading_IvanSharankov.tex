\documentclass[11pt]{article}

\usepackage[english]{babel}
\usepackage{amssymb, bm, caption, csquotes, enumitem, fancyhdr, textcomp, textgreek, gensymb, geometry, graphicx, hyperref, mathrsfs, mathtools, multicol, physics, wrapfig, xcolor}
\usepackage[makeroom]{cancel}

\usepackage{hyperref}
\hypersetup{
    colorlinks=true,
    linkcolor=blue,
    urlcolor=cyan,
}

\newcommand{\threestars}{\begin{center}$ {\ast}\,{\ast}\,{\ast} $\end{center}}

%\renewcommand{\familydefault}{\lmdefault}
%\usepackage{lmodern}
% !TeX spellcheck = en_US 

\geometry{left = 2.54cm, right = 2.54cm, bottom = 1.5cm, top = 2.0cm}
\linespread{1.2}

\pagestyle{fancy}
\fancyhf{}

\rhead{Phys 275 }
\lhead{\thepage}
\chead{Group 19}

\begin{document}
	\title{\textbf{Phys 275 Group Assignment 1}}
	\date{\today}
	\author{Group 19}
	\maketitle
	
	
	\newpage
	\section{Summary}

Section 2.2.2 discussed two types of orbits that can be observed in a 3 body system. In the case where there is one large body with two smaller secondaries orbiting, we can observe what are called 'horseshoe' and 'tadpole' orbits if the semimajor axes of the secondaries are close enough. These orbits are observed in the reference frame rotating along with one of the secondaries and with the origin at the primary body. These orbits are created by the interaction between the two secondaries as they exchange energy on approach. Horseshoe orbits occur when the moving secondary (in the aforementioned reference frame) loses energy when it approaches the other secondary from a higher orbit and gains energy when it approaches from a lower one. This causes it to respectively move to a lower or higher orbit, which it remains in until the next approach. This creates a path that encircles the L3, L4, and L5 lagrangian points - which looks like a horseshoe. If instead this orbit only encompasses the L4 or L5 point, it forms a tadpole orbit.

By observing the intensity of radiation coming from a body across different wavelengths, the resulting spectrum can tell us a lot not only about the chemical composition of the body, but also about the temperature of the body, and the makeup of its atmosphere. For bodies in our solar system, the process is as follows. The sun, as an extremely hot body, emits a continuous thermal spectrum (close to the blackbody spectrum for ~5800k). As this spectrum passes through the lower density outer layers of the sun, certain wavelengths are absorbed, creating the fraunhofer lines. This spectrum then must pass through the atmosphere (if any) of the body we are observing, reflect off its surface, and come through our atmosphere, which adds its own absorption lines. By ignoring the lines created by the sun and our own atmosphere, we might see additional absorption created by a certain material on a planet's surface. By integrating the spectrum over all frequencies, we can find its effective temperature, which is the temperature of a blackbody that would emit the same energy. If we compare this value to the equilibrium temperature we might be able to determine that the body has an internal heating source or is experiencing a strong greenhouse effect.

The basics of the ideal gas law and the barometric law is good to fully understand the relationship both laws share with each other, and can provide us with valuable information about the pressure gradient on a planet. The textbook expanded on this topic by fully explaining how the barometric law can be derived from the ideal gas law by using derivatives and defining the scale height with
\begin{equation}
   H(z) = \frac{kT(z)}{g_p(z)\mu_a(z)m_{\text{amu}}}
\end{equation}

where $g_p(z)$ is the acceleration of gravity at altitude $z$, and $\mu_am_{\text{amu}}$ the molecular mass. This value tells us how quickly an atmosphere's pressure decreases with increasing altitude. In addition, it was stated that if the parameters $g_p$ (acceleration of gravity) $T$ (temperature), and $H$ (scale height, the distance over pressure decreasing by a factor of $e$) vary with increase/decrease of altitude, then the barometric needs to change to accommodate the difference.

	\newpage
	\section{Further Investigation}
\begin{quote}
    “I would also be interested in seeing some more case studies about how the various effects covered in these sections interact with each other. For example, the Io-Europa-Jupiter system was described in 2.7.3 where the horseshoe effect combined with tidal effects caused Io to heat up internally.”
\end{quote}

One interesting example is the Epimetheus-Janus-Saturn system. Epithemeus and Janus orbit Saturn and reach a minimum separation of about 15,000km\footnote{\url{https://solarsystem.nasa.gov/resources/13103/the-dancing-moons/}} (their semimajor axes are only ~50km away from each other). As a result, Epimetheus forms a horseshoe orbit viewed from Janus. Due to the timing and other properties of the system (they are also similar in mass and size), this results in the two moons effectively swapping orbits every 4 years. In this, they are unique in the solar system. The corresponding 7:6 resonance location for Janus maintains a sharp edge to Saturn’s A-Ring, and thus the periodic orbit swap keeps that ring intact, as Janus alternatingly sharpens the inner and outer edges\footnote{\url{https://arxiv.org/abs/1510.00434}}. In this system, we see the effects of resonance, lagrange points, and horseshoe orbits combine to form a visible relic in the rings of Saturn.

This phenomena, where a moon's resonance confines the extent of a planetary ring, is called shepherding, and the moon is known as a shepherd moon. §13.4.3 of Lissauer discusses the phenomenon in some detail, noting plenty of other examples of this, such as Cordelia and Ophelia around Uranus, and Prometheus and Pandora around Saturn. In fact, §13.4 discusses ring-moon interactions in general, where the concepts of resonance, orbital systems (3 body systems in particular), and tidal effects are used to explain the structure of planetary rings.

\threestars

\begin{quote}
    “When it comes to the two restrictions to Planck’s Radiation Law, the text did not give examples regarding what would happen if either of these two limits were exceeded. I wanted to know what kind of data would cause both these limitations to trigger thus providing more information regarding blackbody radiation.”
\end{quote}

The two limitations of Planck’s Radiation Law are Rayleigh-Jeans Law, when $hv << kT$ and Wien’s Law, where $hv >> kT$. When it comes to Wien's Law, it provides us the relationship\footnote{\url{https://aklectures.com/lecture/introduction-to-quantum-theory/wiens-law-example}} between wavelength and absolute temperature that emits radiation. Regarding Rayleigh-Jeans Law, it gives the relationship\footnote{\url{https://scienceworld.wolfram.com/physics/Rayleigh-JeansLaw.html}} between the intensity of radiation and absolute temperature. In terms of Planck’s Law and Rayleigh-Jeans Law, the latter gives us a better approximation on the correlation between $B_v(T)$ and $v(Hz)$. Furthermore, given that Planck’s law is similar to R-J law, it can be noted that the limit is approaching 0, however even with the increase of temperature the function cannot actually be equal to 0.

	\newpage
	\section{Question}
	
\begin{quote}
    “I knew that the heat absorption of a planet equally balanced the heat loss for planets (otherwise we'd burn up or freeze). I was not aware that the gas giants in our solar system were still radiating much more heat than they were absorbing. I wonder if this is often the case with gas giants, or if our gas giants simply have not reached their equilibrium. If it's the latter, assuming the planets were made at relatively the same time (a big give-or-take here), why does it take the gas giants so much more time to stabilize and reach this equilibrium than the rocky inner planets? Does this have something to do with the properties of the gasses in the gas giants? Or maybe the mixing within these planets? It surprises me that the gas giants have not yet reached equilibrium, and I feel there's something missing in the test for why this is the case. What is this reason?”
\end{quote}

The textbook states: “The giant planets Jupiter, Saturn and Neptune are exceptions [to the rule that most planetary bodies radiate the same amount of energy as they absorb from the sun]. These bodies radiate significantly more energy than they absorb because they are still losing heat produced during the time of their formation.” (Lissauer, 88). Seeing as these planets are larger than the terrestrial planets, we might expect that they are able to radiate more power due to the increased surface size, and therefore reach equilibrium faster. The fact that they do not suggests that they might also have a much more powerful internal heat source, which makes sense since the larger size also entails a larger and hotter core.

This is confirmed later in the textbook: “Jupiter, Saturn and Neptune each emit roughly twice as much energy as they receive from the Sun. The excess heat escaping from these planets is attributed to a slow cooling of the planets since their formation combined with energy release by He differentiation (§6.2.3)” (Lissauer, 117). Later, we find that “Helium differentiation is important to a lesser extent for Jupiter.” (Lissauer, 156), so it is likely that the larger size is the main factor for Jupiter’s ongoing cooling.

Extrapolating, we can hypothesize that for a sufficient size and distance from the star, giant planets will continue to cool longer than smaller, solid planets closer to the star. This can also be explained mathematically by noting that the cooling process, radiation from the surface, is a function of the surface area over which this radiation can occur, i.e. $R^2$, while the heating processes (Helium differentiation, radioactive decay, heat of formation, etc.) are functions of the volume of the planet, i.e. of $R^3$.

	\newpage
	\section{Chapter Problems}
	\begin{enumerate}
		\item  \textbf{[Q 2.1] A baseball pitcher can throw a fastball at a speed of ~ 150 km/hr. What is the largest size spherical asteroid of density $\rho = 3000$ kg $\text{m}^{-3}$ from which he can throw the ball fast enough that it:} 
		\begin{enumerate}
			\item  \textbf{escapes from the asteroid into heliocentric orbit?}
			
			A velocity of 150 km/h is equivalent to $41.66\bar{6}$ m/s. Also not that $M = \rho V = \rho \frac{4}{3} \pi r^3$. We find the Kinetic Energy by:
			\[ \text{KE} = \frac{1}{2} mv^2 \]
			And the potential energy as 
			\[ \text{PE} = \frac{GMm}{r} \]
			Equating these we find:
			\begin{align*}
				\frac{1}{2} mv^2 &= \frac{GMm}{r} \\
				\frac{1}{2} v^2 &= \frac{G \left( \rho \cdot \frac{4}{3} \pi r^3 \right)}{r} \\
				v^2 &=  \frac{8}{3} G \rho \pi r^2 \\
				r^2 &= \frac{3 v^2}{8 G \rho \pi} \\
				\therefore r &= \sqrt{\frac{3 v^2}{8 G \rho \pi}} 
			\end{align*}
			\begin{align*}
				r &= \sqrt{\frac{3}{8}\frac{41.666^2}{G \cdot 3000 \cdot \pi}} \\
				r &= 32 287 \text{ m} \\
				\therefore r &= 32.3 \ \text{km}
			\end{align*}
			\item  \textbf{rises to a height of 50 km?}

			The total initial energy is
			\[ \text{E}_i = \frac{1}{2}mv_0^2 - \frac{GMm}{r} \]
			The ball has $v$=0 m/s at the top of the trajectory, so the energy at a height of $h$=50km is
			\[ \text{E}_f = -\frac{GMm}{r + h} \]
			equating these (by conservation of energy),
			\begin{align*}
			    \frac{1}{2}mv_0^2 - \frac{GMm}{r} &= -\frac{GMm}{r+h}  \\
			    \frac{1}{2}v_0^2 - \frac{GM}{r} &= -\frac{GM}{r+h} \\
			    \frac{v_0^2}{2G} &= M\left(\frac{1}{r} - \frac{1}{r+h}\right) \\
			    \frac{v_0^2}{2G} &= \frac{4}{3}\pi r^3 \left(\frac{1}{r} - \frac{1}{r+h}\right)\\
			    \frac{3v_0^2}{8G\pi} &= r^3 \left(\frac{h}{r(r+h)}\right) \\
			    \frac{3v_0^2}{8G\pi} &= \frac{hr^2}{r+h} \\
			    0 &= hr^2 - \frac{3v_0^2}{8G\pi} r - \frac{3v_0^2}{8G\pi}h
			\end{align*}
			plugging in $h=50$m, $v_0 = 41.\bar{6}$m/s and using the quadratic formula, we get $r = 62,153$km
			
			\item  \textbf{goes into a stable orbit about the asteroid?}

		\end{enumerate}
		
		
		
		
		\newpage 
		\item  \textbf{ [Q 2-7] A small planet travels about a star with an orbital semimajor axis equal to 1 in the units chosen. Calculate the locations of the 2:1, 3:2, 99:98 and 100:99 resonances of test particles with the planet.}
		
		This can be solved using only Kepler's Third law, which tells us 
		\[P^2 = \frac{4\pi^2}{GM} a^3 \]
		Where $P$ is the orbital period, $a$ the semimajor axis, and $M$ the mass of the planet.\\ Computing the resonances of the test particles with planet can be done by simply taking the ratio of Kepler's Third law for the two given locations of the bodies, giving us:
		\[ \frac{P_1^2}{P_2^2} = \frac{a_1^3}{a_2^3} \]
		Since we chose the orbital semimajor axis to be 1, we can equate that to $a_2 = 1$, which we can rearrange for $a_1$ as follows:
		
		\begin{align*}
			\frac{P_1^2}{P_2^2} &= \frac{a_1^3}{1^3} \\
			\frac{P_1^2}{P_2^2} &= a_1^3 \\
			\therefore a_1 &= \sqrt[\uproot{3}\scriptstyle 3]{\frac{P_1^2}{P_2^2}}
		\end{align*}
		
		Now we can simply plug in the ratios of $\displaystyle \frac{P_1}{P_2}$. Note: this is the ratio of revolutions the test particle will do in reference with the planet. 2:1 $\approx$ 1.58 tells us during 1 planet rotation, the particle does 1.58 rotations.
		
		\begin{align*}
			\text{for 2:1} \quad \longrightarrow \quad a_1 &= \sqrt[\uproot{3}\scriptstyle 3]{\frac{2^2}{1^2}} \; = \sqrt[\uproot{3}\scriptstyle 3]{\frac{4}{1}} \hspace{26px} = 1.58740105 \\ 
			%
			\text{for 3:2} \quad \longrightarrow \quad a_1 &= \sqrt[\uproot{3}\scriptstyle 3]{\frac{3^2}{2^2}} \; = \sqrt[\uproot{3}\scriptstyle 3]{\frac{9}{4}} \hspace{26px} = 1.31037070 \\ 
			%
			\text{for 99:98} \quad \longrightarrow \quad a_1 &= \sqrt[\uproot{3}\scriptstyle 3]{\frac{99^2}{98^2}} = \sqrt[\uproot{3}\scriptstyle 3]{\frac{9801}{9604}} \hspace{10px} = 1.00338986 \\ 
			%
			\text{for 100:99} \quad \longrightarrow \quad a_1 &= \sqrt[\uproot{3}\scriptstyle 3]{\frac{100^2}{99^2}} = \sqrt[\uproot{3}\scriptstyle 3]{\frac{10000}{9801}} = 1.0033557\\ 
			%
		\end{align*}
		
		
		
		\newpage
		\item [Q 2.12] \textbf{If the Earth-Moon distance was reduced to half its current value then: }
		\begin{enumerate}
			\item \textbf{Neglecting solar tides, how many times as large as at present would the maximum tide heights on Earth be?}
			
			We will start by assuming the average distance between the Earth and Moon, not at apogee and perigee. The average distance between the Earth and Moon is $3.84402\times 10^8$. We will also assume that the height of the bulge is linearly proportional to the force applied (so we need only compare the tidal force in both scenarios). We must first find the current average tidal forces exerted by the Moon. To do this, we can recall that:
			\[ F = \frac{GMm}{r^2} \]
			We can find the tidal force by taking the derivative of this and multiplying it by $-R$:
			\[ F_T = -R \dv{F}{r} = \frac{2RGMm}{r^3} \]
			
			Where R is the radius of the Earth, M is the mass of the Earth, m is the mass of the moon, and r is the distance the moon is at. With this, we can sub in the current mood distance to find:
			\begin{align*}
				F_T &= \frac{2GRMm}{r^3} \\
				&= \frac{2G \times 6378137 \times 5.9736 \times 10^{24} \times 7.3477 \times 10^{22}}{(3.844 \times 10^8)^3} \\
				&= 6.5315 \times 10^{18} \text{ N}
			\end{align*}
			
			Now we must simply do this again, but with a Earth-Moon distance of half the current value. This comes out to $1.922 \times 10^8$ meters. Repeating, we find:
			\begin{align*}
			F_T &= \frac{2GRMm}{r^3} \\
			&= \frac{2G \times 6378137 \times 5.9736 \times 10^{24} \times 7.3477 \times 10^{22}}{(1.922 \times 10^8)^3} \\
			&= 5.2251 \times 10^{19} \text{ N}
			\end{align*}
			
			The difference between these two distances is roughly 12.42\%. At first this seems much to small for an increase when halfing the Earth-Moon distance, but we must recall that since we have $\frac{1}{r^3}$, we should expect such an outcome. Recall:
			\begin{align*}
				\frac{F_\frac{1}{2}}{F_1} = \left( \frac{3.844 \times 10^8}{1.922 \times 10^8}\right)^3 = 2^3 = 8
			\end{align*}
			Therefore $F_1 = \frac{1}{8} F_\frac{1}{2}$, or 0.125, which is very close to out 12.4\% found above.
			
			\item  \textbf{Including solar tides, how many times as large as at present would the maximum tide heights on Earth be?}
			 
			We find the solar tides to be:
			\begin{align*}
				F_T &= \frac{2GRMm}{r^3} \\
				&= \frac{2G \times 6378137 \times 1.989 \times 10^{30} \times 5.9736 \times 10^{24}}{(1.4959 \times 10^{11})^3} \\
				&= 3.000119 \times 10^{18} \text{ N}
			\end{align*}
			Adding this value to both $T_F$'s found above gets us $9.5315 \times 10{18}$ for the current Earth-Moon distance, and $5.5251\times 10^{19}$ N for the halfed Earth-Moon distance. The difference between these two forces accounting for solar tides is then 17.25\%.
			
		\end{enumerate}
	
	\newpage
	\item \textbf{[Q 4.4] Sedna is a minor planet that orbits far beyond the orbit of Pluto. It is difficult to measure the radius of a small, distant Solar System object because optical imaging only constrains $AR^2$ , the product of the square of the radius and the albedo (so for a given brightness, you do not know if the object is big but dark or small and bright). However, Sedna was also observed with the Spitzer Space Telescope, which measured the total flux emitted by Sedna in the infrared. Explain, using the appropriate equation, why this measurement allows the radius to be measured.}
	
	Since the Spitzer Space telescope measured the total flux emitted by Sedna in the infrared, we can find the total flux using the stephan boltzmann equation, $F = \sigma T^4$. Once we have this, we can then easily find the radius of Sedna with the provided constrains of $AR^2$
	
	
	\end{enumerate}
\end{document}
