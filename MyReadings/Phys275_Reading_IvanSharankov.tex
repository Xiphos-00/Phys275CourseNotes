\documentclass[10pt]{article}

\usepackage[english]{babel}
\usepackage{amssymb, bm, caption, csquotes, enumitem, fancyhdr, textcomp, textgreek, gensymb, geometry, graphicx, hyperref, mathrsfs, mathtools, multicol, physics, wrapfig, xcolor}
\usepackage[makeroom]{cancel}

%\renewcommand{\familydefault}{\lmdefault}
%\usepackage{lmodern}
% !TeX spellcheck = en_US 

\geometry{left = 0.5cm, right = 1cm, bottom = 1.5cm, top = 2.0cm}
\linespread{1.2}

\pagestyle{fancy}
\fancyhf{}

\rhead{Ivan Sharankov}
\lhead{20665807}
\chead{Phys 275 Reading 2}

\begin{document}
	
	
	
	
	\subsection*{Three things learned}
	\begin{itemize}
		\item I was not aware that the ratio between incident and reflected + scattered energy could be measured, also known as the monochromatic albedo. The Bond angle is then integrated version over frequency, and seems to be very useful for understanding how much energy a body absorbs, which can tell us the equilibrium temperature of the body.
		
		\item I was not aware there were so many different types of temperature. I've heard of effective temperature before, but not brightness temperature or equilibrium temperature. Brightness temperature seems to be often related to the flux density of the black body, and is preferred since black bodies usually not perfect black bodies. 
			
		\item I didn't know that we can tell a lot about the surface temperature of a distant exoplanet if we find that the spectra of the planet is optically thick in the infrared wavelength. This suggests that the temperature is very high on the planet without directly measuring it, and can tell us that there is a runaway greenhouse effect on the planet.
	\end{itemize}
	
	
	\subsection*{Two things I want to know more about}
	\begin{itemize}
		\item The start of section 4.1 tells us that planets are heated primarily by absorbing radiation from the sun, and they primarily lose energy via radiation. This surprised me since I thought the internal pressure at the core of a planet was a major factor to the heat production of a planet. I thought the internal pressure and heat transfer (from magma on earth, for example) provided a fair portion of the planetary heat. I can see this not being the case, but how about further planets? I cannot believe that Uranus and Neptune are primarily heated by the sun? At such a far distance? I would expect them to be heated by the high pressure in their internal core. With that in mind, I'd like to know more about the ratio of heat between these two sources as a function of the solar system planets (or distance, I assume). 
		
		\item Figure 4.6: I've seen this energy level of hydrogen graph before, so it's not too new. But what I've always wondered is why a different person is named after each wavelength? Seriously. It's the same procedure for the Lyman, Balmer, Paschen, and Brackett series. It's simply the transition energy levels at different wavelengths. Whoever discovered the "first" series, why didn't/couldn't they discover the rest? I find it hard to imagine Balmer finding the ionizing energies for the visible spectrum but never wondered about the infrared. Was it maybe harder and not possible with current technologies? Or maybe they were discovered at the same time independently, so names were given accordingly. What's up with this?
		
		
	\end{itemize}
	
	
	
	\subsection*{Create one question not answered in the text}
	I knew that the heat absorption of a planet equally balanced the heat loss for planets (otherwise we'd burn up or freeze). I was not aware that the gas giants in our solar system were still radiating much more heat than they were absorbing. I wonder if this is often the case with gas giants, or if our gas giants simply have not reached their equilibrium. If it's the latter, assuming the planets were made at relatively the same time (a big give-or-take here), why does it take the gas giants so much more time to stabilize and reach this equilibrium than the rocky inner planets? Does this have something to do with the properties of the gasses in the gas giants? Or maybe the mixing within these planets? It surprises me that the gas giants have not yet reached equilibrium, and I feel there's something missing in the test for why this is the case. What is this reason?
\end{document}